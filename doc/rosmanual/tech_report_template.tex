%%%%%%%%%%%%%%%%%%%%%%%%%%%%%%%%%%%%%%%%%%%%%%%%%%%%%%%%%%%%%%%%%%%%
\documentclass[twoside,10pt]{article}
\usepackage{palatino}
%\usepackage{multirow}
\usepackage{graphicx}
\usepackage{amsmath,amssymb,amsthm}
\usepackage[pdftex=true,colorlinks=true,linkcolor=blue,citecolor=blue,pdfstartview=FitH,naturalnames]{hyperref}

\newcommand{\clearemptydoublepage}{\newpage{\pagestyle{empty}\cleardoublepage}}

\renewcommand{\textfraction}{0.1}
\renewcommand{\topfraction}{0.9}
\renewcommand{\floatpagefraction}{0.8}

% The margins are set for 10pt font.  
\setlength{\oddsidemargin}{0.25in}
\setlength{\evensidemargin}{0.25in}
\setlength{\textwidth}{6in}

% header
\usepackage{fancyhdr}
\pagestyle{fancy}
\fancyhf{}  % clear all
\fancyhead[RE]{\slshape \leftmark}   % slshape = italics
\fancyhead[LO]{\slshape \rightmark}
\fancyhead[LE]{\thepage}
\fancyhead[RO]{\thepage}

\begin{document}

%%% Title Page %%%
% Display front page
% we don't want the date displayed
\date{}

\title{\vspace{1cm}{\huge ROS Manual}}

\author{Morgan Quigley, Ken Conley, Eric Berger, Brian Gerkey, The Wiki \\
\texttt{mquigley@cs.stanford.edu, \{kwc,berger,gerkey\}@willowgarage.com}}

\maketitle
\thispagestyle{empty}

\begin{center}

\vspace{1cm}

{\large July 2009}

\vspace{3cm}

\begin{figure}[!h]
  \centering
  \includegraphics[width=0.27\linewidth,viewport= 10 10 140 90, clip]{WG_logo_on_white}
\end{figure}

\vspace{0.25cm} {\large
  Willow Garage, Inc.\\
  Menlo Park, California 94025\\
}

\vspace{1cm}

\copyright \/ Willow Garage, Inc.

\end{center}




%%% Abstract and Table of Contents %%%
\pagenumbering{Roman}
\clearemptydoublepage
\setcounter{page}{1}

% Display abstract
\begin{This Manual}
This manual provides an overview of ROS (Robot Operating System), its tools, and related software.
\end{This Manual}

% Table of contents 
\clearemptydoublepage
\tableofcontents

\clearemptydoublepage
\pagenumbering{arabic}
\setcounter{page}{1}


\section{Preface}

This manual was written for ROS version 0.7.

\subsubsection{Contents of this Manual}

\subsubsection{ROS Online Resources}

\subsubsection{Acknowledgements}



\section{ROS Overview}

\subsection{What is ROS?}

   * Design (peer-to-peer distributed, favors greater computation power over multiple computers)
   * Not Realtime

\subsubsection{ROS Goals}

   * Important of BSD + Open Source

http://pr.willowgarage.com/wiki/ROS/Overview|h1=Goals

\subsubsection{Key Benefits of ROS}

text from blog post
   * Comparisons to other frameworks
   * Compatibility with other frameworks

\subsubsection{Client Libraries}

ref:http://pr.willowgarage.com/wiki/ROS/client_libs|h1=ROS client libraries



\subsubsection{Concepts}

 * Three levels of ROS: filesystem, computation graph, community

ref:http://pr.willowgarage.com/wiki/ROS/Overview|h1=Concepts


\section{ROS Filesystem Level}

\subsection{Packages}

Packages have dependencies

Types of dependencies: dependencies on Packages, dependencies on thirdparty (rosdep).


\subsubsection{Manifests}

\subsubsection{Package Tools}

       * rospack
       * roscreate-pkg
       * roscd

\subsection{Stacks}


\subsubsection{Message definitions}

     * versioning
   * Service definitions
     * versioning
   * Build System
     * Based on CMake
       * CMake macros
       * Exporting cmake macros
     * Tools
       * make vs. rosmake
       
 * Computation Graph Level
   * Nodes
     * Tools
       * rosnode
     * client libraries
       * rospy.init_node()
       * roscpp node handle
   * Services
     * Tools   
       * rosservice
     * client libraries
      * rospy.ServiceProxy
       * roscpp Service
   * Topics
     * Transports
       * TCPROS
       * UDPROS
     * Tools       
       * rostopic
       * rxplot
     * client libraries
       * Example with roscpp
       * No example with rospy
   * Bags
     * Tools
       * rosrecord
       * rosplay
     * Client libraries
       * rosrecord.py
   * The Graph
     * The ROS Core
       * Master
       * Parameter Server
         * rosparam
       * rosout
         * rxconsole
         * log files
     * roslaunch
     * run_id

 * Community Level
   * Repository
     * Repos
       * ROS
       * ros-pkg
       * wg-ros-pkg
     * Tools
       * roslocate
   * Mailing lists
   * ROS Wiki
   * Bug tracker
   * Blog

Using ROS
 * Getting comfortable with the command line
   * roscreate-pkg foo
   * create MyMsg.msg
   * edit CMakeLists.txt file
   * run rosmake, as well as make
   * run roscore
 * Writing Nodes
   * roscpp example
     * Simple Talker/Listener
   * rospy example
     * Simple Talker/Listener
   * Wrap up your nodes in roslaunch
     * Pass in parameter
   * Services
     * roscpp
       * Edit talker to provide a service
       * Edit listener to call a service     
     * rospy
       * Edit talker to provide a service
       * Edit listener to call a service     


\section{Our Solution to the Robot Vision Problem}\label{sec:solution}


\subsection{Minor Points}


\section{Results}


\section{There's More}

The ROS wiki at http://ros.org/wiki is a living document that includes
many more tutorials and descriptions of additional software you can
use with ROS.

\clearpage


\end{document}

